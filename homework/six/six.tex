\input{../physics_header.tex}
\usepackage{amsfonts, amsmath, amsthm}

% a matter of taste
\setlength{\parskip}{1ex}
\setlength{\parindent}{0pt}

\newtheorem*{exer}{Exercise}

% A few simple macros for group theory.
\newcommand{\floor}{\text{floor }}
\newcommand{\img}{\text{img }}
\newcommand{\lcm}{\text{lcm }}
\newcommand{\aut}{\text{Aut }}
\newcommand{\cycle}[1]{(\mathbf{#1})}

\begin{document}

\textbf{Homework 6 -- Quantum Mechanics} \\

\hrule

% Problem list
\begin{minipage}{.80\linewidth}
    \flushleft
    Ch 3: 1, 2, 4, 16 \\
\end{minipage}
\begin{minipage}{.20\linewidth}
    \flushright
    % whoami
    Blake Griffith
\end{minipage}

% % % % % % % % % % % % % % % % % % % % % % % % % % % % % % % % % % % % 
% Problems
% % % % % % % % % % % % % % % % % % % % % % % % % % % % % % % % % % % % 

\begin{exer}[3.1]

    Find the eigenvalues and eigenvectors of 
    $\sigma_y = \begin{pmatrix}
        0 & -i \\ 
        i & 0
    \end{pmatrix}$
    . Suppose an electron is in the spin state $
    \twobyone{\alpha}{\beta} $. If $S_y$ is measured, what is the
    probability of the result $\hbar / 2$?


\end{exer}

\begin{proof}

    \begin{enumerate}
        \item We begin by solving the charachteristic equation to find
            the eigenvalues of $\sigma_y$:
            \[
            \begin{vmatrix}
                -\lambda & -i \\
                i & -\lambda
            \end{vmatrix} = \lambda^2 -  1 = 0 \implies
            \boxed{\lambda = \pm 1}
            \]
            Now we find the corresponding eigenvectors, for $\lambda =
            -1$:
            \[
            \sigma_y \twobyone{a}{b} = \twobyone{-a}{-b} \implies 
            -ib = -a \qquad ai = -b \implies
            \boxed{\v{X}(\lambda = -1) =
            \frac{1}{\sqrt{2}}\twobyone{1}{-i}}
            \]
            for $\lambda = 1$:
            \[
            \sigma_y \twobyone{a}{b} = \twobyone{a}{b} \implies 
            -ib = a \qquad ai = b \implies
            \boxed{\v{X}(\lambda = 1) =
            \frac{1}{\sqrt{2}}\twobyone{1}{i}}
            \]
            
        \item
            The probability of measuring $S_y = \hbar/2$ (spin up in the
            $y$ basis) is $\left|\bra{\psi_{y+}} S_y
            \ket{\psi}\right|^2$. Where is $\bra{\psi_{y+}}$ is the
            eigenstate corresponding to the
            $S_y$ ``up'' eigenvalue ($\lambda = 1$). Which we found
            above since $S_y = \frac{\hbar}{2}\sigma_y$. So we have
            \[
                \left| \bra{\psi_{y+}} S_y \ket{\psi} \right|^2 \implies
                \frac{\hbar^2}{8} \left|
                    \begin{pmatrix} 1 & i \end{pmatrix}
                    \begin{pmatrix} 0 & -i \\ i & 0 \end{pmatrix}
                    \twobyone{\alpha}{\beta} \right|^2 \implies 
                    \boxed{\frac{\hbar^2}{8}(\alpha^2 + \beta^2)}
            \]

    \end{enumerate}


\end{proof}

% % % % % % % % % % % % % % % % % % % % % % % % % % % % % % % % % % % % 

\begin{exer}[3.2]

    Find by explicit construction using Pauli matrices, the eigenvalues
    for the Hamiltonian
    \[
        H = -\frac{2\mu}{\hbar} \v{S} \cdot \v{B} 
    \]
    for a spin $\frac{1}{2}$ particle in the presence of a magnetic
    field $\v{B} = B_x \hat{\v{x}} + B_y \hat{\v{y}} + B_z \hat{\v{z}}$.

\end{exer}

\begin{proof}

\end{proof}

% % % % % % % % % % % % % % % % % % % % % % % % % % % % % % % % % % % % 

\begin{exer}[3.4]

    The spin-dependent Hamiltonian of an electron-positron system in the
    presence of a uniform magnetic field in the $z$-direction can be
    written as

    \[
        H = A \v{S}^{e^-} \cdot \v{S^{e^+}} + \frac{e B}{mc}
        \left(S^{e^-}_{z} - S^{e^+}_{z} \right)
    \]

    Suppose the spin function of the system is given by $\chi^{e^-}_+
    \chi^{e^+}_-$.
    \begin{enumerate}
        \item Is this an eigenfunction of $H$ in the limit $A \to
            0$, $eB/mc \not= 0$? If it is, what is the energy eigenvalue?
            If it is not, what is the expectation value of $H$?
        \item Solve the same problem when $eB/mc \to 0$, $A \not= 0$.
    \end{enumerate}

\end{exer}

\begin{proof}

\end{proof}

% % % % % % % % % % % % % % % % % % % % % % % % % % % % % % % % % % % % 

\begin{exer}[3.16]

    Show that the orbital angular-momentum operator $\v{L}$ commutes
    with both the operators $\v{p}^2$ and $\v{x}^2$; that is, prove
    (3.7.2).

\end{exer}

\begin{proof}

\end{proof}

\end{document}
